\section{Sajber kriminal i napadi}

\begin{frame}{Pojam sajber kriminala}
        
        \begin{itemize}
        	\item \textit visokotehnološki ili sajber kriminal (eng. cyber crime)
        	\begin{itemize} 
	        	\item moderni vid kriminala
    	    	\item putem mreže 
        		\item izvođenje zlonamernih aktivnosti
            \end{itemize}
			\item ciljevi:
				\begin{itemize}
					\item krađa osetljivih podataka neke firme
					\item krađa ličnih informacija 
					\item profit
            	\end{itemize}
        \end{itemize}
    \end{frame}
    
    
   \begin{frame}{Zakon u Srbiji}
    \begin{itemize}
    \item globalni indeks sajber bezbednosti (eng. GCI) - pouzdan izvor koji meri posvećenost država bezbednosti na Internetu 
    \item Srbija je zemlja sa niskom stopom posvećenosti suzbijanju sajber kriminala
    \end{itemize}
	\begin{minipage}{0.3\textwidth}
        \includegraphics[scale = 0.25]{itu_worldmap.jpg}
        \end{minipage}
        \begin{minipage}{0.6\textwidth}\raggedleft
        Tamnozelena-najviše posvećen\\
        crvena-najmaje posvećen\\
        \end{minipage}
     
    \end{frame}
    
    \begin{frame}{Sajber napadi i vrste}
    \begin{itemize}
	\item \textit sajber napad - napad od računara do računara 
	\item povređuje poverljivost, integritet i informacije koje se nalaze na napadnutom računaru  
	\item sa porastom popularnosti Interneta porasla je i stopa kriminala na njemu
	\item najčešće vrste napada
	\begin{itemize}
	\item Fišing (eng. phishing)
	\item SQL injekcija
	\item DoS napadi
\end{itemize}	  
    \end{itemize}
    \end{frame}
    
    %\frame{\sectionpage}
    
    \begin{frame}{Fišing (eng. \textit{phishing})}
        
        \noindent\begin{minipage}{0.55\linewidth}
            \begin{itemize}
                \item lažno predstavljanje pomoću elektronske pošte ili zlonamernih veb sajtova kako bi se prikupili lični podaci
                \item termin phishing: phreaks + fishing
                \item mejlovi obaveštavaju primaoce da je njihov nalog kompromitovan
                \item ciljano orijentisan fising \\(eng. spear phishing)
            \end{itemize}
        \end{minipage}
        \hfill
        \begin{minipage}[t]{0.4\linewidth}
            \centering
            \includegraphics[scale = 0.3]{images/phishing1.jpeg}
            \vspace{0.5cm}
            \includegraphics[scale = 0.3]{images/spear_phishing_transparent.png}
        \end{minipage}
        %\includegraphics[scale = 0.3]{spear_phishing_transparent.png}
    \end{frame}
    
    \begin{frame}{Fišing (eng. \textit{phishing})}
        \begin{minipage}{0.6\textwidth}
        \begin{itemize}
                \item fišing se pojavio negde oko 1995. (AOL)
                \item 2000ih godina su napadi prešli na platne sisteme (E-Gold, eBay i PayPal)
            \end{itemize}
        \end{minipage}
        \begin{minipage}[t]{0.3\linewidth}
            \centering
            \includegraphics[scale = 0.045]{images/aol_logo.jpg}
            \hspace{0.1cm}
            \includegraphics[scale = 0.05]{images/PayPal_eBay.jpg}
        \end{minipage}
        
        \begin{minipage}{\textwidth}
            \centering
            \vspace{0.3cm}
            \includegraphics[width=0.7\textwidth, height = 5cm]{images/phishing.png}
        \end{minipage}
        \begin{minipage}{\textwidth}
            \centering
            \textit{broj sajtova za fišing koje je APWG otkrio tokom godina}
        \end{minipage}
    \end{frame}

\begin{frame}{SQL injekcija}
        
        %\noindent\begin{minipage}{0.6\linewidth}
            \begin{itemize}
                \item umetanje dela ili celog SQL upita obično preko polja za unos na veb stranici
                \item načini upotrebe SQL injekcije:
                		\begin{itemize}
                		\item \texttt{1=1} je uvek istinito
                		        \includegraphics[width=\linewidth]{sql1.png}
                		\item \texttt{"}\texttt{"}=\texttt{"}\texttt{"} je uvek istinito
                		        \includegraphics[width=\linewidth]{sql2.png}                		        
                		\item grupa SQL upita razdvojenih ; simbolom
                		        \includegraphics[width=\linewidth]{sql3.png}
                		\end{itemize}
                	\item zaštita od ovakvih napada je moguća korišćenjem SQL parametara
            \end{itemize}
    \end{frame}


\begin{frame}{DoS napadi (eng. \textit{Denial-of-Service})}

	\begin{itemize}

		\item napadač - ometa usluge servera i čini ga nedostupnim svojim korisnicima
		\item asimetričan napad - jedna osoba može dosta da naškodi velikoj organizaciji
		\item cilj - remećenje sposobnosti servera, a ne krađa informacija
		\item dodatni tip DDoS (eng. \textit{Distributed Denial-of-Service}) - meta je napadnuta sa više lokacija odjednom
		\item zaštita - razviti plan odgovora na napad, osigurati mrežnu infrastrukturu, održavati jaku arhitekturu mreže, razumeti znakove upozorenja

	\end{itemize}

\end{frame}


\begin{frame}{Primeri napada}

	\noindent\begin{minipage}{0.55\linewidth}
	%\vspace{1cm}
    \begin{itemize}

    \item samostalni napadi
    	\begin{itemize}
    		\item Dženson Džejms Ančeta
    		\item PharmaMaster
    	\end{itemize}
    
    \item napadi na države
    	\begin{itemize}
    		\item Gruzija
    		\item SAD i Južna Koreja
    	\end{itemize}
    	
    \item napadi na kompanije
    	\begin{itemize}
    		\item Yahoo!
    		\item eBay
    	\end{itemize}
    \end{itemize}
        \end{minipage}
        
        \hfill
        \begin{minipage}[t]{0.5\linewidth}
            \centering
	        \vspace{-4cm}
            \includegraphics[scale = 0.2]{images/napad.png}
        \end{minipage}
    
    \end{frame}
    

\begin{frame}{Sajber kriminal}

	\begin{itemize}
		
		\item zlonameran k\^{o}d (eng. \textit{malware}) - glavno sredstvo za trgovinu informacijama i infiltriranje u sisteme
		\item napadači - jako pažljivi i prikrivaju svoje tragove
		\item botnet-ovi - podstiču rast sajber napada i dozvoljavaju napadaču da kontroliše neki sistem širom Interneta
		
	\end{itemize}

\end{frame}