% !TEX encoding = UTF-8 Unicode
\documentclass[a4paper]{article}

\usepackage{color}
\usepackage{url}
\usepackage[T2A]{fontenc} % enable Cyrillic fonts
\usepackage[utf8]{inputenc} % make weird characters work
\usepackage{graphicx}

\usepackage[english,serbian]{babel}
%\usepackage[english,serbianc]{babel} %ukljuciti babel sa ovim opcijama, umesto gornjim, ukoliko se koristi cirilica

\usepackage[unicode]{hyperref}
\hypersetup{colorlinks,citecolor=green,filecolor=green,linkcolor=blue,urlcolor=blue}

\usepackage{listings}

%\newtheorem{primer}{Пример}[section] %ćirilični primer
\newtheorem{primer}{Primer}[section]

\definecolor{mygreen}{rgb}{0,0.6,0}
\definecolor{mygray}{rgb}{0.5,0.5,0.5}
\definecolor{mymauve}{rgb}{0.58,0,0.82}

\lstset{ 
  backgroundcolor=\color{white},   % choose the background color; you must add \usepackage{color} or \usepackage{xcolor}; should come as last argument
  basicstyle=\footnotesize,        % the size of the fonts that are used for the code
  breakatwhitespace=false,         % sets if automatic breaks should only happen at whitespace
  breaklines=true,                 % sets automatic line breaking
  captionpos=b,                    % sets the caption-position to bottom
  commentstyle=\color{mygreen},    % comment style
  deletekeywords={...},            % if you want to delete keywords from the given language
  escapeinside={\%*}{*)},          % if you want to add LaTeX within your code
  extendedchars=true,              % lets you use non-ASCII characters; for 8-bits encodings only, does not work with UTF-8
  firstnumber=1000,                % start line enumeration with line 1000
  frame=single,	                   % adds a frame around the code
  keepspaces=true,                 % keeps spaces in text, useful for keeping indentation of code (possibly needs columns=flexible)
  keywordstyle=\color{blue},       % keyword style
  language=Python,                 % the language of the code
  morekeywords={*,...},            % if you want to add more keywords to the set
  numbers=left,                    % where to put the line-numbers; possible values are (none, left, right)
  numbersep=5pt,                   % how far the line-numbers are from the code
  numberstyle=\tiny\color{mygray}, % the style that is used for the line-numbers
  rulecolor=\color{black},         % if not set, the frame-color may be changed on line-breaks within not-black text (e.g. comments (green here))
  showspaces=false,                % show spaces everywhere adding particular underscores; it overrides 'showstringspaces'
  showstringspaces=false,          % underline spaces within strings only
  showtabs=false,                  % show tabs within strings adding particular underscores
  stepnumber=2,                    % the step between two line-numbers. If it's 1, each line will be numbered
  stringstyle=\color{mymauve},     % string literal style
  tabsize=2,	                   % sets default tabsize to 2 spaces
  title=\lstname                   % show the filename of files included with \lstinputlisting; also try caption instead of title
}

\begin{document}

\title{Bezbednost na internetu: Sajber kriminal i sajber napadi, glasanje preko interneta\\ \small{Seminarski rad u okviru kursa\\Metodologija stručnog i naučnog rada\\ Matematički fakultet}}

\author{Prvi autor, drugi autor, treći autor, četvrti autor\\ kontakt email prvog, drugog, trećeg, četvrtog autora}

%\date{9.~april 2015.}

\maketitle

\abstract{
U ovom tekstu je ukratko prikazana osnovna forma seminarskog rada. Obratite pažnju da je pored ove .pdf datoteke, u prilogu i odgovarajuća .tex datoteka, kao i .bib datoteka korišćena za generisanje literature. Na prvoj strani seminarskog rada su naslov, apstrakt i sadržaj, i to sve mora da stane na prvu stranu! Kako bi Vaš seminarski zadovoljio standarde i očekivanja, koristite uputstva i materijale sa predavanja na temu pisanja seminarskih radova. Ovo je samo šablon koji se odnosi na fizički izgled seminarskog rada (šablon koji \emph{morate} da ispoštujete!) kao i par tehničkih pomoćnih uputstava. Pročitajte tekst pažljivo jer on sadrži i važne informacije vezane za zahteve obima i karakteristika seminarskog rada.}

\tableofcontents

\newpage

\section{Uvod}
\label{sec:uvod}

%Kada budete predavali seminarski rad, imenujete datoteke tako da sadrže redni broj teme, temu seminarskog rada, kao i prezimena članova grupe. Precizna uputstva na temu imenovnja će biti data na formi za predaju seminarskog rada. Predaja seminarskih radova biće isključivo preko veb forme, a NE slanjem mejla. Link na formu će biti dat u okviru obaveštenja na strani kursa. Vodite računa da prilikom predavanja seminarskog rada predate samo one fajlove koji su neophodni za ponovno generisanje pdf datoteke. To znači da pomoćne fajlove, kao što su .log, .out, .blg, .toc, .aux i slično, \textbf{ne treba predavati}.

%\section{Osnovna uputstva}
%Vaš seminarski rad mora da sadrži najmanje jednu \textbf{sliku}, najmanje jednu \textbf{tabelu} i najmanje \textbf{sedam referenci} u spisku literature. Najmanje jedna slika treba da bude originalna i da predstavlja neke podatke koje ste Vi osmislili da treba da prezentujete u svom radu. Isto važi i za najmanje jednu tabelu. 	Od referenci, neophodno je imati bar jednu \textbf{knjigu}, bar jedan \textbf{naučni članak} iz odgovarajućeg časopisa i bar jednu adekvatnu \textbf{veb adresu}. 

%\textbf{Dužina seminarskog rada treba da bude od 10 do 12 strana.}

%Ко жели, може да пише рад ћирилицом. У том случају, неопходно је да су инсталирани одговарајући пакети: texlive-fonts-extra, texlive-latex-extra, texlive-lang-cyrillic, texlive-lang-other. 

%Nemojte koristiti stari način pisanja slova, tj ovo:
%\begin{verbatim}
%\v{s} i \v{c} i \'c ...
%\end{verbatim}
%Koristite direknto naša slova:	
%\begin{verbatim}
%š i č i ć ... 
%\end{verbatim}


%\section{Engleski termini i citiranje}	
%\label{sec:termini_i_citiranje}

%Na svakom mestu u tekstu naglasiti odakle tačno potiču informacije. Uz sve novouvedene termine u zagradi naglasiti od koje engleske reči termin potiče. 

%Naredni primeri ilustruju način uvođenja enlegskih termina kao i citiranje.

%\begin{primer}
%Problem zaustavljanja (eng.~{\em halting problem}) je neodlučiv \cite{haltingproblem}.
%\end{primer}

%\begin{primer}
%Za prevođenje programa napisanih u programskom jeziku C može se koristiti GCC kompajler \cite{gcc}.
%\end{primer}

%\begin{primer}
% Da bi se ispitivala ispravost softvera, najpre je potrebno precizno definisati njegovo ponašanje \cite{laski2009software}. 
%\end{primer}

%Reference koje se koriste u ovom tekstu zadate su u datoteci {\em seminarski.bib}. Prevođenje u pdf format u Linux okruženju može se uraditi na sledeći način:
%\begin{verbatim}
%pdflatex TemaImePrezime.tex 
%bibtex TemaImePrezime.aux 
%pdflatex TemaImePrezime.tex 
%pdflatex TemaImePrezime.tex 
%\end{verbatim}
%Prvo latexovanje je neophodno da bi se generisao {\em .aux} fajl. {\em bibtex} proizvodi odgovarajući %{\em .bbl} fajl koji se koristi za generisanje literature. 
%Potrebna su dva prolaza (dva puta pdflatex) da bi se reference ubacile u tekst (tj da ne bi ostali znakovi pitanja umesto referenci). Dodavanjem novih referenci potrebno je ponoviti ceo postupak.  


Napomena: U uvodnom delu treba imati više od dva citata. Milena je na predavanju rekla da treba što više citata da se ubacuje. Postoji dva tipa citata koje je preporuceno koristiti:\\
1. gde se može naći detaljnije o toj temi\\
2. objašnjenja. Tipa odakle tvrdim da je to važno, koje istraživanje je ivršeno, pa to iznosim kao činjenicu...\\
Ovde \cite{knjiga} citiram knjigu po kojoj smo izabrali poglavlja








%Broj naslova i podnaslova je proizvoljan. Neophodni su samo Uvod i Zaključak. Na poglavlja unutar teksta referisati se po potrebi. 
%\begin{primer}
%U odeljku \ref{sec:naslov1} precizirani su osnovni pojmovi, dok su zaključci dati u odeljku \ref{sec:zakljucak}.
%\end{primer}

%Još jednom da napomenem da nema razloga da pišete:
%\begin{verbatim}
%\v{s} i \v{c} i \'c ...
%\end{verbatim}
%Možete koristiti srpska slova
%\begin{verbatim}
%š i č i ć ... 
%\end{verbatim}



%\section{Slike i tabele}
%\label{slike_i_tabele}

%Slike i tabele treba da budu u svom okruženju, sa odgovarajućim naslovima, obeležene labelom da koje omogućava referenciranje. 

%\begin{primer} Ovako se ubacuje slika. Obratiti pažnju da je dodato i 
%\begin{verbatim}
%\usepackage{graphicx}
%\end{verbatim}

%\begin{figure}[h!]
%\begin{center}
%\includegraphics[scale=0.75]{panda.jpg}
%\end{center}
%\caption{Pande}
%\label{fig:pande}
%\end{figure}

%Na svaku sliku neophodno je referisati se negde u tekstu. Na primer, na slici \ref{fig:pande} prikazane su pande. 
%\end{primer}

%\begin{primer} I tabele treba da budu u svom okruženju, i na njih je neophodno referisati se u tekstu. Na primer, u tabeli \ref{tab:tabela1} su prikazana različita poravnanja u tabelama.

%\begin{table}[h!]
%\begin{center}
%\caption{Razlčita poravnanja u okviru iste tabele ne treba koristiti jer su nepregledna.}
%\begin{tabular}{|c|l|r|} \hline
%centralno poravnanje& levo poravnanje& desno poravnanje\\ \hline
%a &b&c\\ \hline
%d &e&f\\ \hline
%\end{tabular}
%\label{tab:tabela1}
%\end{center}
%\end{table}

%\end{primer}

%\section{K\^{o}d i paket listings}
%Za ubacivanje koda koristite paket \textbf{listings}:
%\url{https://en.wikibooks.org/wiki/LaTeX/Source_Code_Listings}

%Primer ubacivanja koda
%\begin{lstlisting}[frame=single]
%# This program adds up integers in the command line
%import sys
%try:
%    total = sum(int(arg) for arg in sys.argv[1:])
%    print 'sum =', total
%except ValueError:
%    print 'Please supply integer arguments'
%\end{lstlisting}


\section{Pojam sajber kriminala}
\label{pojam}

Visokotehnološki ili sajber kriminal (eng. {\em cyber criminal}) predstavlja moderni vid kriminala, tačnije, putem računara. Sajber kriminalci su osobe ili grupe ljudi koji koriste tehnologiju kako bi izveli zlonamerne aktivnosti putem mreže sa ciljem da ukradu osetljive podatke neke firme, lične informacije ili da profitiraju \cite{sajber}.
\\Zakoni koji se odnose na ovu vrstu kriminala se dopunjuju i razvijaju u zemljama širom sveta. Najizloženije zemlje za sajber napade su one koje su u razvoju. U takvim zemljama je zakon o ovoj oblasti slabo definisan, a u nekim ni ne postoji. Takođe, veoma je teško pronaći i uhapsiti zločinca u sajber kriminalu jer su dokazi često nepostojeći.

Treba napraviti razliku između sajber kriminalca i hakera. Sajber kriminalci sa lošim namerama vrše upad u računare, dok hakeri traže inovativne načine da koriste sistem, bili ti načini loši ili dobri.
\\

 

\section{Sajber kriminal i napadi}

\label{sec:kriminal_napadi}

U mnogim zemljama, internet odvija ključnu ulogu u svakodnevnom životu ljudi. Olakšava virtuelnu komunikaciju među ljudima, podstiče razvoj novih poslovnih modela i kompanija, menja način na koji ljudi kupuju. U 2018. godini, transfer novca koji uključuje prodaju i kupovinu putem interneta, iznosila je oko 2800 milijardi američkih dolara. Prema statističkim studijama \cite{statistika}, smatra se da će u 2021. ta vrednost iznositi oko 4.88 biliona dolara. Iz ovih razloga, nije neobično da je sa porastom popularnosti interneta porasla i stopa kriminala na njemu. U ovom poglavlju otkrivamo tri vrste napada putem interneta.

\subsection{Phishing}
\label{subsec:phishing}

US-CERT (The United States Computer Emergency Readiness Team) definiše “phishing” kao vrstu “social engineering”-a gde se napadač pomoću elektronske pošte ili zlonamernih veb sajtova lažno predstavlja kao pouzdana organizacija ili kompanija kako bi prikupio lične podatke od pojedinca ili kompanije. Napadi “phishing”-a se često sastoje od slanja korisnicima imejlova koji izgledaju kao da su iz bankarske ili finansijske institucije ili veb servisa preko kojeg pojedinac ima račun. Cilj “phishing”-a je da prevari primaoca da da svoje podatke za prijavljivanje ili druge osetljive informacije. 
\\Na primer, napadač može da pošalje milione imejlova sa botnet-a. Poruke obaveštavaju primaoce da je njihov nalog za elektronsku trgovinu bio kompromitovan i upućuju ih na veb lokaciju gde bi rešili problem. Korisnici koji kliknu na link dođu do veb stranice koja je napravljena tako da podseća na originalni sajt za elektronsku trgovinu. Kada se nađu na sajtu, od njih se traži korisničko ime, lozinka i druge privatne informacije. Te informacije mogu da se iskoriste za krađu identiteta.

Ciljani (spear) “phishing” je varijanta “phishing”-a u kojoj napadač bira adrese elektonske pošte tako da cilja jednog ili određenu grupu primalaca. Na primer, napadač može ciljati starije osobe kao osobe koje se smatraju lakovjernijima ili članove grupa
koji imaju pristup vrednim informacijama. “Spear phishing” može biti veoma delotvoran jer omogućava napadaču da uobliči napad tako da žrtva zbog hitnosti ili poverenja određenim osobama bude manje oprezna. Za “spear phishing” je potrebno da napadač prikupi lične podatke o žrtvi, kao što su imena prijatelja, poslodavac, rodni grad, lokacije koje posećuje, šta je nedavno kupila na mreži...
Na primer, napadač može da pošalje imejl nekoliko ljudi koji izgleda kao da je od njihovog direktora, gde im je poslat poziv na sastanak putem Gmaila, a link u poruci navodi primaoce da se prijave na Gmail da prisustvuju sastanku. 
 
Prema jednom istraživanju, bilo je najmanje 67.000 “phishing” napada širom sveta u drugoj polovini 2010. godine. Zanimljivo
je povećanje “phishing” napada na kineske e-trgovine, što ukazuje na povećavanje važnosti kineske ekonomije. U 2018 godini
APWG (Anti-Phishing Working Group) je otkrio 785 920 sajtova za “phishing”.

\subsection{SQL injekcija}
\label{subsec:sql}

SQL injekcija jeste umetanje dela ili celog SQL upita obično preko polja za unos na veb stranici. Ukoliko ovako nešto uspe može se pristupiti osetljivim podacima iz baze, mogu se modifikovati podaci, izvršiti administrativne operacije nad bazom itd. Pogledajmo primer ispod koji kreira SELECT upit koji dodaje sadržaj promenljive(txtUserId) na SELECT string. Sadržaj promenljive je sadržaj polja za unos korisničkog id-a (getRequestString).
\begin{verbatim}
txtUserId = getRequestString("UserId");
txtSQL = "SELECT * FROM Users WHERE UserId = " + txtUserId;
\end{verbatim}

\paragraph{}
Jedan od načina upotrebe SQL injekcije zasniva se na činjenici da je ,,1=1‘‘ uvek istinito. Zamislimo da je korisnik u polju za unos uneo ,,105 OR 1=1‘‘ . Tada bi SQL upit iz prethodnog primera izgledao ovako:

\begin{verbatim}
SELECT * FROM Users WHERE UserId = 105 OR 1=1;
\end{verbatim}


\noindent Ovakav upit vratiće sve redove ,,Users‘‘ tabele, jer je ,,1=1‘‘ uvek istinito. Šta ako tabela ,,Users‘‘ sadrzi imena i šifre? Haker moze pristupiti svim imenima i šiframa iz baze jednostavno dodavajuci ,,105 OR 1=1‘‘ u polje za unos korisničkog imena.

\paragraph{}
 ,, \texttt{"}\texttt{"}=\texttt{"}\texttt{"} ‘‘ je uvek istinito i ovo je još jedan način upotrebe SQL injekcije. Recimo da imamo sledeći deo koda:
\begin{verbatim}
uName = getRequestString("username");
uPass = getRequestString("userpassword");

sql = 'SELECT * FROM Users WHERE Name ="' + uName + '" AND 
								Pass ="' + uPass + '"'
\end{verbatim}

\noindent Haker jednostavno može pristupiti korisničkim imenima i šiframa u bazi unoseći ,, \texttt{"} OR \texttt{"}\texttt{"}=\texttt{"} ‘‘ u polje za šifru ili u polje za korisničko ime. Kod na serveru ce kreirati ispravan SQL upit:

\begin{verbatim}
SELECT * FROM Users WHERE Name ="" OR ""="" AND Pass ="" OR ""="";
\end{verbatim}

\noindent SQL upit koji se nalazi iznad vratiće sve redove iz tabele ,,Users‘‘, jer je ,, OR \texttt{"}\texttt{"}=\texttt{"}\texttt{"} ‘‘ uvek istinito. 

\paragraph{}
Mnoge baze podrzavaju grupu SQL upita razdvojene ,, ; ‘‘. SQL upit ispod vratiće sve redove iz tabele ,,Users‘‘, i potom obrisati ,,Suppliers‘‘ tabelu.

\begin{verbatim}
SELECT * FROM Users; DROP TABLE Suppliers
\end{verbatim}

\noindent Pogledajmo sledeći primer:

\begin{verbatim}
txtUserId = getRequestString("UserId");
txtSQL = "SELECT * FROM Users WHERE UserId = " + txtUserId;
\end{verbatim}

\noindent Ukoliko bi korisnik u polje za korisnički id uneo ,,105; DROP TABLE Suppliers‘‘ SQL upit koji se nalazi iznad izgledao bi ovako:

\begin{verbatim}
SELECT * FROM Users WHERE UserId = 105; DROP TABLE Suppliers; 
\end{verbatim}

\paragraph{}
Kako se zaštititi od ovakvih napada? Tako što cemo koristiti SQL parametre. SQL parametri su vredosti koje su dodate SQL upitu u vreme izvršavanja na kontrolisan način. 

\begin{verbatim}
txtUserId = getRequestString("UserId");
txtSQL = "SELECT * FROM Users WHERE UserId = @0";
db.Execute(txtSQL,txtUserId);
\end{verbatim}

\noindent Primer iznad je deo koda u ASP.NET-u u kome se koriste parametri. Parametri su predstavljeni znakom @. SQL mehanizam proverava parametre kako bi se uverio da su ispravni i da se tretiraju bukvalno a ne kao deo SQL-a koji se izvršava.


\subsection{DoS napadi}
\label{subsec:DoS}

https://www.paloaltonetworks.com/cyberpedia/what-is-a-denial-of-service-attack-dos

DoS (Denial-of-Service) napad je radnja napravljena tako da spreči legitimitne korisnike da koriste usluge računara, tj. počinilac (izvršilac) čini mašinu ili mrežni resurs nedostupnim (gasi ih) svojim korisnicima tako što privremeno ili neograničeno ometa usluge hosta povezanog na Internet. DoS napad moze da uključi neovlašćeni pristup jednom ili više kompjuterskih sistema, ali cilj napada nije krađa informacija, nego je cilj da poremeti sposobnost servera da odgovori na korisničke zahteve tako što poplavljuje metu saobraćajem (?? popravi) ili šalje informacije koje aktiviraju razne nezgode. Ometanje normalnog rada kopjuterskih usluga moze da proizvede značajnu stetu. Firma koja se bavi nekom vrstom prodaje putem Interneta može da izgubi posao. Vojsci može da se prekine komunikacija. Vladi ili nekoj neprofitnoj organizaciji može da se desi da ne može da prenese svoju poruku javnosti.

DoS napad je primer "asimetričnog" napada, u kome jedna osoba moze dosta da naškodi velikoj organizaciji. Pošto se terorističke organizacije specijalizuju za asimetrične napade, neki strahuju da će DoS napadi postati važan deo terorističkog oružja.

\subsubsection{DDoS napadi}
\label{subsubsec:DDoS}

Dodatni tip DoS napada je DDoS (Distributed Denial-of-Service) napad. Glavna razlika je u tome što meta nije napadnuta sa jedne lokacije, već sa više njih odjednom. Do DDoS napada se dolazi kada višestruki sistemi orkestriraju sinhronizovani DoS napad na jednu metu. Podela hostova koji određuju DDoS daje napadaču više prednosti:
\begin{itemize}
\item Napadač može iskoristiti veću količinu mašine (popravi) da izvrši ozbiljno razoran napad
\item Lokacija napada se teško određuje zbog slučajne podele napadačkih sistema
\item Teže je ugasiti više mašina nego jednu
\item Pravu napadačku partiju (attacking party - potrazi) je veoma teško identifikovati, jer se oni prikrivaju iza mnogih (uglavnom kompromitovanih) sistema
\end{itemize}

Mnoge sigurnosne tehnologije su razvile mehanizme za odbranu od mnogih vrsta DoS napada, ali, zbog jedinstvenih karakteristika, DDoS se još uvek smatra ozbiljnom pretnjom.

\section{Sajber kriminal}
\label{sec:sajber_kriminal}

http://docshare04.docshare.tips/files/23471/234718626.pdf \\
deo Advancement of  Computer and Network Attack and Exploitation Methodologies (163-164)


Glavni trend koji pokreće evoluciju računarskih napada i ekspoatacija uključuje motiv rasta profita zasnovanog na zlonamernom kodu (malware, malicious code). Neki napadači prodaju prilagodjeni zlonamerni kod za kontrolu mašina žrtava kupcu koji je najviše ponudio. Mogu iznajmiti gomile zaraženih sistema, koji su korisni za isporuku nepoželjne pošte (spamova), phishing shema, DoS napada ili za krađu identiteta. Špijunske kompanije i preterano agresivni oglašivači (popravi) kupuju takvu vrstu koda kako bi se ubacili (infiltrirali) i kontrolisali žrtvine mašine. Jedna zaražena mašina na kojoj iskašu reklame i prilagođavaju se rezultati pretrage može da košta svega \$1 mesešno. Ključni logger (???) na zaraženoj mašini može da pomogne napadaču da skupi brojeve kreditnih kartica i ukrade \$1,000 ili više od žrtve pre nego što se otkrije prevara. Ako kontroliše 10,000 mašina, napadač može solidno da profitira od sajber kriminala. Organizovane kriminalne grupe mogu okupiti grupu takvih napadaša kako bi stvorile biznis, što izaziva porast industrije zlonamernog koda. Krajem devedesetih godina proslog veka, vecinu malvera su pravili entuzijasti, ali danas su profesionalni napadaši monetizovali svoj malver, njihovi profitni centri daju sredstva koja se mogu koristiti za istraživanje i razvoj kako bi se stvorio snažan zlonameran softver i ometali poslovni modeli, kao i finansirali drugi zločini. 

Kada kriminalci budu otkrili pouzdan nacin za zaradu novca na ovoj grani kriminala, broj incidenata ovog tipa će se neizbežno povecati. Kompjuterski napadači su osmislili različite poslovne modele koji su niskog rizika, tako da je verovatnoca da će napadač biti uhvacen ukoliko pazljživo skriva svoje tragove mala. Tako mogu da se zarade stotine hiljada ili čak milioni dolara.

Faktor koji podstiče rast sajber napada je bot (skracenica od reci \textit{robot}) softver. Ovaj softver dozvoljava napadaču da kontroliše neki sistem širom Interneta. Jedan napadač ili grupa napadača moze da postavi ogromne botnetove (grupe zaraženih mašina) širom sveta. Mašine koje kontrolišu botovi omogućavaju napadačoma da postavljaju virtualne superračunare koji mogu da predstavljaju rivale nacionalnoj kompjuterskoj snazi. Mogu da koriste te resurse da bi stvorili ogromnu poplavu, razbili (crack) kripto ključeve ili lozinke, ili da bi istražili osetljive finansijske podatke koji se koriste za krađu identiteta.

Botovi i ostali alati koji se koriste za napade su postali veoma modularni, svaki modul se sastoji iz softverskih komponenti koje omogućavaju napadačima da brzo menjaju funkcionalnost kako bi pokrenuli nove vrste napada. Obični botovi danas sadrže 50 do 100 različitih funkcionalnih modula; napadač može da ugasi ili odrstani module koji nisu potrebni za dati napad, dok lako integriše nove karakteristike (funkcije) koda. Drugi modularni napadački alati sadrže ekspoatacione okvire (exploitation  frameworks) koji prave pakovani ekspoatacioni kod koji može da se ubaci (infiltrira) u ciljnu mašinu koja je ranjiva (zato pto je pogrepno konfigurisana ili nepovezana).

Ubrzavajući evoluciju, napadači se sve više oslanjaju na bot kod koji se sam preobražava, dinamički stvarajući funkcionalno ekvivalentnu verziju sa različitim skupovima osnovnog koda. Takav polimorfni kod pomaze napadačima da izbegnu alate za detekciju koje antivirusi i antispyware softveri\footnote{Vrsta programa napravljena za sprečavanje i otkrivanje neželjenih instalacija špijunskih softvera i uklanjanje tih programa ukoliko su instalirani.} danas koriste. Ovaj dinamički samopodešavajući kod je teze filtrirati, s obzirom da konstantno modulira (menja) svoj osnovi softver. Ova "pokretna meta" koda otežava analizu od stane branilaca. Polimorfni kod podstiče ciljeve napadača jer napadači imaju dužu kontrolu nad botnetom izbegavajući filtriranje i detekcije, tako da mogu što više novca da zarade od zaraženih sistema.

\subsection{Primeri kriminalnih napada}
\label{subsec:primeri_krimi_napada}

\begin{description}
\item[DŽENSON DŽJEMS ANČETA] 2004. i 2005. je Dženson Džejms Ančeta, radnik u jednom Internet kafeu, napravio mrežu od oko 400,000 botova, uključujući i računare kojima upravlja Ministarstvo odbrane SAD-a. Softveri za reklame (adware companies), spameri i ostali su platili Ančeti da koriste te računare. Nakon što ga je FBI uhapsio, Ančeta se izjasnio krivim za razne optužbe, uključujući prekršavanje zakona o zlostavljanju računara kao i CAN-SPAM zakon (Computer Fraud Abuse Act and the CAN-SPAM Act). U maju 2005., federalni sudija je osudio Ančetu na 57 meseci zatvora i tražio od njega da plati \$15,000 američkoj vladi zbog napada na Odeljenje za odbranu. Zbog njegovih ilegalinih aktivnosti, vlada je Ančeti oduzela njegov 1993 BMW, više od \$60,000 u kešu, kao i njegovu računarsku opremu.

\item[PHARMAMASTER] Izraelska kompanija Blue Security je napravila sistem za zastrašivanje spamova (popravi !!!) kako bi pomogla ljudima koji ne žele da primaju spampve. Blue Security je prodavala svoj sistem preduzećima, dok su individualci mogli besplatno da štite svoje računare. Oko pola miliona ljudi se prijavilo za ovu besplatni uslugu. Korisnici su na svojim računarima učitali bot po nazivu Blue Frog koji je integrisan sa Yahoo! Mail-om, Gmail-om, Hotmail-om, i provera da li su dolazeći mejlovi spamovi. Kada naiđe na spam poruku, bot kontaktira Blue Security server kako bi otkrio izvor tog mejla i zatim spameru salje opt-out poruku (da prekine da salje takvu vrstu mejlova).

Spameri koji neselektivno salju mejlove na milione adresa su počeli da primaju stotine hiljada opt-out poruka, koje su ometale njihove operacije. Šest najboljih svetskih spamera se dogovorilo da koristi softver za filtriranje koji je razvio Blue Security kako bi odstranili Blue Frog korisnike sa svojih mejl lista.

Jedan spamer, čiji je nadimak bio PharmaMaster, se nije povukao. On je pretio Blue Frog korisnicima porukama kao sto je na primer sledeća : "Nažalost, zbog taktika koje koristi Blue Security, primaćete ovu poruku ili druge besmislene spamove 20-40 puta vise nego inače". On je ispoštovao svoje pretnje, pa je 1. maja 2006. počeo da šalje Blue Frog korisnicima 10 do 20 puta više spamova nego sto obično prime.

Narednog dana je počeo da napada sam Blue Security. Lansirao je DDoS napad sa desetina hiljada botova ciljajući Blue Security servere. Ogromna bujica (torrent) dolaznih poruka je onemogucila Blue Frog uslugu. Kasnije su se DDoS napadi fokusirali na druge kompanije koje pružaju Internet usluge Blue Security-u. Na kraju su napadi ciljali preduzeća koja su placala usluge Blue Seciruty-u. Kada je Blue Security shvatio da ne može da zaštiti svoje klijente od DDoS napada i mejlova sa virusima, nevoljno je prekinuo svoju uslugu.

\item[AVALANCHE GANG] Avalanche Gang je ime dato kriminalnom preduzeću koje je odgovorno za više phishing napada nego bilo koja druga organizacija. Anti-Phishing Working Group (APWG) je procenila da je Avalanche Gang odgovoran za dve trećine svih phishing napada lansiranih u drugoj polovini 2009. U drugoj polovini 2010., APWG je zapazila da je Avalanche skoro prestao sa phishing napadima, vodeci APWG da spekuliše da je Avalanche menjao strategije i fokusirao se na širenje neželjene pošte koja ljude uvlači u preuzimanje Zeus trojanskog konja.
\item[ALBERT GONZALEZ] Albert Gonzalez je 2010. osuđen na 20 godina zatvora nakon što se izjasnio krivim za koriscenje SQL injekcija kako bi ukrao više od 130 miliona brojeva kreditnih i debitnih kartica. Neki od tih brojeva su prodati onlajn, što je dovelo do neovlašćenih kupovina i povlačenja banaka. Ciljevi napada su bili Heartland Payment Systems,  7-Eleven,  Hannaford Brothers Supermarkets,  TJX,  DSW,  Barnes \& Noble, OfficeMax, and the Dave \& Buster lanci restorana. Većina brojeva je ukradena od firme Heartland Payment Systems, procenjujući gubitak na \$130 miliona.
\end{description}

\section{Sajber napadi}
\label{sec:sajber_napadi}

Sajber napad predstavlja napad od računara do računara koji povređuje poverljivost, integritet i informacije koje se nalaze na napdanutom računaru \cite{knjiga}. U ovom poglavlju ćemo se susresti sa nekim primerima napada na države i velike svetske gigante u poslednjih deset godina. 

\subsection{Primeri napada na države}
\label{subsec:primeri_napada_drzave}

\begin{enumerate}
\item GRUZIJA (2008) 
\\Gruzija je jedna od bivših Sovjetskih republika koja je stekla nezavisnost 1991. gidine. Južna Osetija, oblast na teritoriji Gruzije koja je pripadala Rusiji do 1991, nakon kratkog rata iste godine, postaje i međunarodno priznata kao autonomna pokrajna Gruzije.  Nakon provokacije separatista u Južnoj Osetiji, Gruzija šalje vojsku na ovu teritoriju 7. avgusta 2008. godine. Ruske snage su 8. avgusta ušle u Južnu Osetiju i ove dve strane su se borile četiri dana. Ovaj sukob je ostao zapamćen i iz razloga što je gruzijska vlada i pre nego što je ruska vojska došla na teritoriju Južne Osetije imala problem sa velikim DDoS napadom. Njihova vlada nije bila u mogućnosti da komunicira sa ostatkom sveta. Mnogi veb sajtovi su bili srušeni na nekoliko sati. Gruzijska vlada je bila primorana da lokacije nekih svojih veb servera prebaci na SAD. Postojale su sumnje je da je napad izvrvršila grupa kriminalaca pod nazivom "Russian Business Network", smeštenih Sent Peterburgu u Rusiji, ali je ostalo nerazlučeno da li je ova grupa imala neke veze sa ruskom vojskom.
\\Na isti dan godinu dana kasnije, Tviter je bio onesposobljen u celom svetu na nekoliko sati zbog masovnih DDoS napada. Maks Keli, šef za bezbednost u kompaniji Fejsbuk, rekao je da je svrha napada bila da se spreči objavljivanje teksta gruzijskog političkog blogera, pozivajući se na činjenicu da su istovremeno pali i ostali sajtovi koje je ovaj bloger koristio. Ti sajtovi su bili Fejsbuk, LiveJournal i Gugl.\\     
\item SAD I JUŽNA KOREJA (2009) 
\\DDos napad na američku i južnokorejsku vladu je izvršen tokom vikenda uoči 4. jula, američkog Dana nezavisnosti. Tom prilikom zabeležen je pad više od trećine veb sajtova u ovim državama. U Americi je napadnuta Bela kuća, američki trezor, Tajna služba, njujoršku berzu i kompaniju NASDAQ. U Južnoj Koreji, DDoS napad je izvršen na Plavu kuću(predsednička palata), Ministarstvo odbrane i Narodnu skupštinu.
\\Ova vrsta napada se smatra relativno malom jer je izvršen uz pomoć botnet-a(naći referencu za botnet) koristeći između 50 i 65 hiljada računara, što se smatra malom cifrom u poređenju na velike napade gde se koristi oko milion računara. Ipak, ova vrsta napada je ostala zapamćena i po tome što su južnokorejski sajtovi ostali nedostupni do 9. jula. Pretpostavljalo se da je napad izvršen kao vid osvete jer su Ujedinjene Nacije uvele određene sankcije Severnoj Koreji u tom periodu. I do danas se još uvek ne zna ko je tačno izveo ovaj napad jer su napadači koristili računare koji su bili u posedstvu drugih ljudi.    

\end{enumerate}


\subsection{Primeri napada na kompanije}
\label{subsec:primeri_napada_kompanije}


\begin{enumerate}
\item Yahoo
\\Yahoo je jedan od najvećih giganta na internetu. Ova kompanija je 2016. godine objavila informaciju da je bila žrtva jednog od najvećih napada u istoriji; 2013. godine je grupa hakera kompromitovala tri milijarde naloga korisnika. Pored imena, datuma rođenja, imejl adresa i šifara, zaštitna pitanja i odgvori su takođe otkriveni. 
Pored ovog napada, 2014. je zabeležen još jedan napad na ovu kompaniju. Tada su obelodanjena imena, imejl adrese, datumi rođenja i brojevi telefona 500 miliona korisnika. Ovoga puta su šifre ostale zaštićene.\\ 
\item eBay
\\Ova kompanija se bavi prodajom proizvoda putem interneta. Bila je napadnuta u maju 2014. kada su otkrivena imena, adrese, datumi rođenja i enkriptovane šifre(naći referencu za enkripciju) od 145 miliona korisnika. Kriminalci su "upali" u bazu tako što su koristili kreditacije troje zaposlenih i imali su pristup unutrašnjosti 229 dana.U tom periodu su imali vremena da pristupe bazi podataka korisnika. Informacije koje se tiču finansija kao što su brojevi kreditnih kartica su ostale zaštićene jer se ta vrsta podataka čuva u odvojenoj bazi.\\
\item Uber
\\Uber je američka kompanija koja se predstavlja kao mreža koja pruža uslue transporta. Ona je 2016. bila napadnuta od strane samo dva hakera koji su uspeli da dođu do imena, imejl adresa i brojeva telefona 57 miliona korisnika Uber aplikacije. Takođe, otkriveni brojevi vozačkih dozvola 600000 vozača na ovoj platformi.\\Hakeri su pristupili i Uberovom nalogu na GitHub platformi(referenca) gde su pronašli korisničko ime i lozinku kreditacije ka AWS nalogu(referenca). 
Uber je ovaj napad objavio godinu dana kasnije. Ova kompanija je platila 100000 američkih dolara hakerima da unište podatke.  
\end{enumerate}


\section{Glasanje preko interneta}
\label{sec:glasanje}

Postoje mnogi načini na koje napadači mogu da prekrše bezbednost umreženih računara. Međutim, praktičnost i niska cena obavljanja poslova putem interneta donose značajne prednosti, pa nije iznenađujuće da se onlajn rešenje često predlaže kada postoji problem sa tradicionalnim načinom. U ovom delu razmotricemo predlog za sprovođenje izbora putem interneta.

\subsection{Motivacija za glasanje putem interneta}
\label{subsec:Motivacija za glasanje putem interneta}

Predsednički izbori 2000. godine su bili jedni od najneizvesnijih u istoriji SAD. Florida je bila država od glavnog značaja. Bez izbornih glasova na Floridi, ni demokrata Al Gor (eng. {\em Al Gore}), ni republikanac Džordž Buš (eng. {\em George V. Bush}) nisu imali većinu glasova. Posle ručnog prebrojavanja glasova u četiri veoma demokratske pokrajine, sekretar države Florida je izjavio da je Buš pobedio sa 2,912,790 glasova u odnosu na 2,912,253 glasova koje je dobio Gor. Bušova prednost bila je neverovatno mala: manje od 2 glasa na svakih 10.000 glasova.

Većina ovih okruga koristila je mašinu za glasanje u kojoj birači biraju kandidata tako što olovkom probuše rupu u kartici pored odgovarajućeg imena (slika dodati). Uočene su dve nepravilnosti u glasanju upotrebom ovih mašina. Prva nepravilnost je da ponekad olovka ne probuši glatko rupu, ostavljajući mali, pravougaoni komad kartice koji visi sa jednog ili više uglova. Takve glasove obično ne uračuna mašina za automatsko prebrojavanje glasova, pa se ručno prebrojavanje fokusiralo na identifikaciji takvih glasačkih listića. Druga nepravilnost bila je da su neki birači u okrugu Palm Beach bili zbunjeni glasačkim listićima i pogrešno probušili rupu koja odgovara kandidatu Patu Bakananu umesto rupe za demokratskog kandidata Al Gor. Ova konfuzija je možda koštala Al Gora glasova koji su mu bili potrebni za pobedu na Floridi.

\subsection{Predlozi}
\label{subsec:Predlozi}

Problemi sa izborima na Floridi doveli su do raznih akcija za poboljšanje pouzdanosti glasačkih sistema u Sjedinjenim Državama. Mnoge države su zamenile papirne sisteme elektronskim glasačkim aparatima za direktno očitavanje. Druge su predložile da se koristi glasanje putem interneta, makar radi odbacivanja korišćenja glasačkih listića. U stvari, onlajn glasanje već postoji. Korišćeno je u Aljasci 2000. godine za republikansku anketu za predsedničkog kandidata i u Arizoni za demokratsku predsedničku listu za  premijera. Na predsedničkim izborima 2004. godine, 100.000 Amerikanaca u vojsci i onih koji žive u inostranstvu je trebalo da ima priliku da glasa preko interneta kao deo eksperimenta za sigurnu elektronsku registraciju i glasanje (eng. {\em “Secure Electronic Registration and Voting Experiment”}), ali se vlada predomislila u poslednjem trenutku.
Mnoge zemlje su ispred Sjedinjenih Država po pitanju uvođenja glasanja preko interneta. Lokalni izbori u Ujedinjenom Kraljevstvu koristili su onlajn glasanje 2001. godine. Građanima koji žive u Sjedinjenim Državama bilo je dozvoljeno da koriste internet da biraju svoje predstavnike u Skupštini francuskih građana u inostranstvu. Estonija je bila prva zemlja koja je omogućila svim svojim građanima da glasaju putem interneta na lokalnim i nacionalnim izborima. Nekoliko kantona u Švajcarskoj je ustavnim promenama odobrilo internet kao zvaničnu opciju za glasanje, pored biračkih mesta i glasanja poštom.

\section{Zaključak}
\label{sec:zakljucak}

Ovde pišem zaključak. 
Ovde pišem zaključak. 
Ovde pišem zaključak. 
Ovde pišem zaključak. 
Ovde pišem zaključak. 
Ovde pišem zaključak. 
Ovde pišem zaključak. 
Ovde pišem zaključak. 
Ovde pišem zaključak. 
Ovde pišem zaključak. 
Ovde pišem zaključak. 
Ovde pišem zaključak.

 
\addcontentsline{toc}{section}{Literatura}

\appendix
\bibliography{seminarski} 
\bibliographystyle{plain}
\begin{enumerate}
\item \url{https://www.trendmicro.com/vinfo/us/security/definition/cybercriminals}
\item \url{https://cybersecurityventures.com/cybercrime-damages-6-trillion-by-2021/} 
\item \url{https://www.paloaltonetworks.com/cyberpedia/what-is-a-denial-of-service-attack-dos}
\item \url{https://www.w3schools.com/sql/sql_injection.asp}
\item \url{https://www.csoonline.com/article/2130877/the-biggest-data-breaches-of-the-21st-century.html}
\item \url{http://docshare04.docshare.tips/files/23471/234718626.pdf} str 163-164
\end{enumerate}

\end{document}
