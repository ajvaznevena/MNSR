% !TEX encoding = UTF-8 Unicode
\documentclass[a4paper]{article}

\usepackage{color}
\usepackage{url}
\usepackage[T2A]{fontenc} % enable Cyrillic fonts
\usepackage[utf8]{inputenc} % make weird characters work
\usepackage{graphicx}

\usepackage[english,serbian]{babel}
%\usepackage[english,serbianc]{babel} %ukljuciti babel sa ovim opcijama, umesto gornjim, ukoliko se koristi cirilica

\usepackage[unicode]{hyperref}
\hypersetup{colorlinks,citecolor=green,filecolor=green,linkcolor=blue,urlcolor=blue}

\usepackage{listings}

%\newtheorem{primer}{Пример}[section] %ćirilični primer
\newtheorem{primer}{Primer}[section]

\definecolor{mygreen}{rgb}{0,0.6,0}
\definecolor{mygray}{rgb}{0.5,0.5,0.5}
\definecolor{mymauve}{rgb}{0.58,0,0.82}

\lstset{ 
  backgroundcolor=\color{white},   % choose the background color; you must add \usepackage{color} or \usepackage{xcolor}; should come as last argument
  basicstyle=\footnotesize,        % the size of the fonts that are used for the code
  breakatwhitespace=false,         % sets if automatic breaks should only happen at whitespace
  breaklines=true,                 % sets automatic line breaking
  captionpos=b,                    % sets the caption-position to bottom
  commentstyle=\color{mygreen},    % comment style
  deletekeywords={...},            % if you want to delete keywords from the given language
  escapeinside={\%*}{*)},          % if you want to add LaTeX within your code
  extendedchars=true,              % lets you use non-ASCII characters; for 8-bits encodings only, does not work with UTF-8
  firstnumber=1000,                % start line enumeration with line 1000
  frame=single,	                   % adds a frame around the code
  keepspaces=true,                 % keeps spaces in text, useful for keeping indentation of code (possibly needs columns=flexible)
  keywordstyle=\color{blue},       % keyword style
  language=Python,                 % the language of the code
  morekeywords={*,...},            % if you want to add more keywords to the set
  numbers=left,                    % where to put the line-numbers; possible values are (none, left, right)
  numbersep=5pt,                   % how far the line-numbers are from the code
  numberstyle=\tiny\color{mygray}, % the style that is used for the line-numbers
  rulecolor=\color{black},         % if not set, the frame-color may be changed on line-breaks within not-black text (e.g. comments (green here))
  showspaces=false,                % show spaces everywhere adding particular underscores; it overrides 'showstringspaces'
  showstringspaces=false,          % underline spaces within strings only
  showtabs=false,                  % show tabs within strings adding particular underscores
  stepnumber=2,                    % the step between two line-numbers. If it's 1, each line will be numbered
  stringstyle=\color{mymauve},     % string literal style
  tabsize=2,	                   % sets default tabsize to 2 spaces
  title=\lstname                   % show the filename of files included with \lstinputlisting; also try caption instead of title
}

\begin{document}

\title{Bezbednost na internetu: Sajber kriminal i sajber napadi, glasanje preko interneta\\ \small{Seminarski rad u okviru kursa\\Metodologija stručnog i naučnog rada\\ Matematički fakultet}}

\author{Prvi autor, drugi autor, treći autor, četvrti autor\\ kontakt email prvog, drugog, trećeg, četvrtog autora}

%\date{9.~april 2015.}

\maketitle

\abstract{
U ovom tekstu je ukratko prikazana osnovna forma seminarskog rada. Obratite pažnju da je pored ove .pdf datoteke, u prilogu i odgovarajuća .tex datoteka, kao i .bib datoteka korišćena za generisanje literature. Na prvoj strani seminarskog rada su naslov, apstrakt i sadržaj, i to sve mora da stane na prvu stranu! Kako bi Vaš seminarski zadovoljio standarde i očekivanja, koristite uputstva i materijale sa predavanja na temu pisanja seminarskih radova. Ovo je samo šablon koji se odnosi na fizički izgled seminarskog rada (šablon koji \emph{morate} da ispoštujete!) kao i par tehničkih pomoćnih uputstava. Pročitajte tekst pažljivo jer on sadrži i važne informacije vezane za zahteve obima i karakteristika seminarskog rada.}

\tableofcontents

\newpage

\section{Uvod}
\label{sec:uvod}

%Kada budete predavali seminarski rad, imenujete datoteke tako da sadrže redni broj teme, temu seminarskog rada, kao i prezimena članova grupe. Precizna uputstva na temu imenovnja će biti data na formi za predaju seminarskog rada. Predaja seminarskih radova biće isključivo preko veb forme, a NE slanjem mejla. Link na formu će biti dat u okviru obaveštenja na strani kursa. Vodite računa da prilikom predavanja seminarskog rada predate samo one fajlove koji su neophodni za ponovno generisanje pdf datoteke. To znači da pomoćne fajlove, kao što su .log, .out, .blg, .toc, .aux i slično, \textbf{ne treba predavati}.

%\section{Osnovna uputstva}
%Vaš seminarski rad mora da sadrži najmanje jednu \textbf{sliku}, najmanje jednu \textbf{tabelu} i najmanje \textbf{sedam referenci} u spisku literature. Najmanje jedna slika treba da bude originalna i da predstavlja neke podatke koje ste Vi osmislili da treba da prezentujete u svom radu. Isto važi i za najmanje jednu tabelu. 	Od referenci, neophodno je imati bar jednu \textbf{knjigu}, bar jedan \textbf{naučni članak} iz odgovarajućeg časopisa i bar jednu adekvatnu \textbf{veb adresu}. 

%\textbf{Dužina seminarskog rada treba da bude od 10 do 12 strana.}

%Ко жели, може да пише рад ћирилицом. У том случају, неопходно је да су инсталирани одговарајући пакети: texlive-fonts-extra, texlive-latex-extra, texlive-lang-cyrillic, texlive-lang-other. 

%Nemojte koristiti stari način pisanja slova, tj ovo:
%\begin{verbatim}
%\v{s} i \v{c} i \'c ...
%\end{verbatim}
%Koristite direknto naša slova:	
%\begin{verbatim}
%š i č i ć ... 
%\end{verbatim}


%\section{Engleski termini i citiranje}	
%\label{sec:termini_i_citiranje}

%Na svakom mestu u tekstu naglasiti odakle tačno potiču informacije. Uz sve novouvedene termine u zagradi naglasiti od koje engleske reči termin potiče. 

%Naredni primeri ilustruju način uvođenja enlegskih termina kao i citiranje.

%\begin{primer}
%Problem zaustavljanja (eng.~{\em halting problem}) je neodlučiv \cite{haltingproblem}.
%\end{primer}

%\begin{primer}
%Za prevođenje programa napisanih u programskom jeziku C može se koristiti GCC kompajler \cite{gcc}.
%\end{primer}

%\begin{primer}
% Da bi se ispitivala ispravost softvera, najpre je potrebno precizno definisati njegovo ponašanje \cite{laski2009software}. 
%\end{primer}

%Reference koje se koriste u ovom tekstu zadate su u datoteci {\em seminarski.bib}. Prevođenje u pdf format u Linux okruženju može se uraditi na sledeći način:
%\begin{verbatim}
%pdflatex TemaImePrezime.tex 
%bibtex TemaImePrezime.aux 
%pdflatex TemaImePrezime.tex 
%pdflatex TemaImePrezime.tex 
%\end{verbatim}
%Prvo latexovanje je neophodno da bi se generisao {\em .aux} fajl. {\em bibtex} proizvodi odgovarajući %{\em .bbl} fajl koji se koristi za generisanje literature. 
%Potrebna su dva prolaza (dva puta pdflatex) da bi se reference ubacile u tekst (tj da ne bi ostali znakovi pitanja umesto referenci). Dodavanjem novih referenci potrebno je ponoviti ceo postupak.  


Napomena: U uvodnom delu treba imati više od dva citata. Milena je na predavanju rekla da treba što više citata da se ubacuje. Postoji dva tipa citata koje je preporuceno koristiti:\\
1. gde se može naći detaljnije o toj temi\\
2. objašnjenja. Tipa odakle tvrdim da je to važno, koje istraživanje je ivršeno, pa to iznosim kao činjenicu...\\
Ovde \cite{knjiga} citiram knjigu po kojoj smo izabrali poglavlja








%Broj naslova i podnaslova je proizvoljan. Neophodni su samo Uvod i Zaključak. Na poglavlja unutar teksta referisati se po potrebi. 
%\begin{primer}
%U odeljku \ref{sec:naslov1} precizirani su osnovni pojmovi, dok su zaključci dati u odeljku \ref{sec:zakljucak}.
%\end{primer}

%Još jednom da napomenem da nema razloga da pišete:
%\begin{verbatim}
%\v{s} i \v{c} i \'c ...
%\end{verbatim}
%Možete koristiti srpska slova
%\begin{verbatim}
%š i č i ć ... 
%\end{verbatim}



%\section{Slike i tabele}
%\label{slike_i_tabele}

%Slike i tabele treba da budu u svom okruženju, sa odgovarajućim naslovima, obeležene labelom da koje omogućava referenciranje. 

%\begin{primer} Ovako se ubacuje slika. Obratiti pažnju da je dodato i 
%\begin{verbatim}
%\usepackage{graphicx}
%\end{verbatim}

%\begin{figure}[h!]
%\begin{center}
%\includegraphics[scale=0.75]{panda.jpg}
%\end{center}
%\caption{Pande}
%\label{fig:pande}
%\end{figure}

%Na svaku sliku neophodno je referisati se negde u tekstu. Na primer, na slici \ref{fig:pande} prikazane su pande. 
%\end{primer}

%\begin{primer} I tabele treba da budu u svom okruženju, i na njih je neophodno referisati se u tekstu. Na primer, u tabeli \ref{tab:tabela1} su prikazana različita poravnanja u tabelama.

%\begin{table}[h!]
%\begin{center}
%\caption{Razlčita poravnanja u okviru iste tabele ne treba koristiti jer su nepregledna.}
%\begin{tabular}{|c|l|r|} \hline
%centralno poravnanje& levo poravnanje& desno poravnanje\\ \hline
%a &b&c\\ \hline
%d &e&f\\ \hline
%\end{tabular}
%\label{tab:tabela1}
%\end{center}
%\end{table}

%\end{primer}

%\section{K\^{o}d i paket listings}
%Za ubacivanje koda koristite paket \textbf{listings}:
%\url{https://en.wikibooks.org/wiki/LaTeX/Source_Code_Listings}

%Primer ubacivanja koda
%\begin{lstlisting}[frame=single]
%# This program adds up integers in the command line
%import sys
%try:
%    total = sum(int(arg) for arg in sys.argv[1:])
%    print 'sum =', total
%except ValueError:
%    print 'Please supply integer arguments'
%\end{lstlisting}


\section{Pojam sajber kriminala}
\label{pojam}

Visokotehnološki ili sajber kriminal (eng. {\em cyber criminal}) predstavlja moderni vid kriminala, tačnije, putem računara. Sajber kriminalci su osobe ili grupe ljudi koji koriste tehnologiju kako bi izveli zlonamerne aktivnosti putem mreže sa ciljem da ukradu osetljive podatke neke firme, lične informacije ili da profitiraju \cite{sajber}.
\\Zakoni koji se odnose na ovu vrstu kriminala se dopunjuju i razvijaju u zemljama širom sveta. Najizloženije zemlje za sajber napade su one koje su u razvoju. U takvim zemljama je zakon o ovoj oblasti slabo definisan, a u nekim ni ne postoji. Takođe, veoma je teško pronaći i uhapsiti zločinca u sajber kriminalu jer su dokazi često nepostojeći.

Treba napraviti razliku između sajber kriminalca i hakera. Sajber kriminalci sa lošim namerama vrše upad u računare, dok hakeri traže inovativne načine da koriste sistem, bili ti načini loši ili dobri.
\\

 

\section{Sajber kriminal i napadi}

\label{sec:kriminal_napadi}

Ovde pišem tekst. 
Ovde pišem tekst. 
Ovde pišem tekst. 
Ovde pišem tekst. 
Ovde pišem tekst. 

\subsection{Phishing}
\label{subsec:phishing}

US-CERT (The United States Computer Emergency Readiness Team) definiše “phishing” kao vrstu “social engineering”-a gde se napadač pomoću elektronske pošte ili zlonamernih veb sajtova lažno predstavlja kao pouzdana organizacija ili kompanija kako bi prikupio lične podatke od pojedinca ili kompanije. Napadi “phishing”-a se često sastoje od slanja korisnicima imejlova koji izgledaju kao da su iz bankarske ili finansijske institucije ili veb servisa preko kojeg pojedinac ima račun. Cilj “phishing”-a je da prevari primaoca da da svoje podatke za prijavljivanje ili druge osetljive informacije. 
\\Na primer, napadač može da pošalje milione imejlova sa botnet-a. Poruke obaveštavaju primaoce da je njihov nalog za elektronsku trgovinu bio kompromitovan i upućuju ih na veb lokaciju gde bi rešili problem. Korisnici koji kliknu na link dođu do veb stranice koja je napravljena tako da podseća na originalni sajt za elektronsku trgovinu. Kada se nađu na sajtu, od njih se traži korisničko ime, lozinka i druge privatne informacije. Te informacije mogu da se iskoriste za krađu identiteta.

Ciljani (spear) “phishing” je varijanta “phishing”-a u kojoj napadač bira adrese elektonske pošte tako da cilja jednog ili određenu grupu primalaca. Na primer, napadač može ciljati starije osobe kao osobe koje se smatraju lakovjernijima ili članove grupa
koji imaju pristup vrednim informacijama. “Spear phishing” može biti veoma delotvoran jer omogućava napadaču da uobliči napad tako da žrtva zbog hitnosti ili poverenja određenim osobama bude manje oprezna. Za “spear phishing” je potrebno da napadač prikupi lične podatke o žrtvi, kao što su imena prijatelja, poslodavac, rodni grad, lokacije koje posećuje, šta je nedavno kupila na mreži...
Na primer, napadač može da pošalje imejl nekoliko ljudi koji izgleda kao da je od njihovog direktora, gde im je poslat poziv na sastanak putem Gmaila, a link u poruci navodi primaoce da se prijave na Gmail da prisustvuju sastanku. 
 
Prema jednom istraživanju, bilo je najmanje 67.000 “phishing” napada širom sveta u drugoj polovini 2010. godine. Zanimljivo
je povećanje “phishing” napada na kineske e-trgovine, što ukazuje na povećavanje važnosti kineske ekonomije. U 2018 godini
APWG (Anti-Phishing Working Group) je otkrio 785 920 sajtova za “phishing”.

\subsection{SQL injekcija}
\label{subsec:sql}

SQL injekcija jeste umetanje dela ili celog SQL upita obično preko polja za unos na veb stranici. Ukoliko ovako nešto uspe može se pristupiti osetljivim podacima iz baze, mogu se modifikovati podaci, izvršiti administrativne operacije nad bazom itd. Pogledajmo primer ispod koji kreira SELECT upit koji dodaje sadržaj promenljive(txtUserId) na SELECT string. Sadržaj promenljive je sadržaj polja za unos korisničkog id-a (getRequestString).
\begin{verbatim}
txtUserId = getRequestString("UserId");
txtSQL = "SELECT * FROM Users WHERE UserId = " + txtUserId;
\end{verbatim}

\paragraph{}
Jedan od načina upotrebe SQL injekcije zasniva se na činjenici da je ,,1=1‘‘ uvek istinito. Zamislimo da je korisnik u polju za unos uneo ,,105 OR 1=1‘‘ . Tada bi SQL upit iz prethodnog primera izgledao ovako:

\begin{verbatim}
SELECT * FROM Users WHERE UserId = 105 OR 1=1;
\end{verbatim}


\noindent Ovakav upit vratiće sve redove ,,Users‘‘ tabele, jer je ,,1=1‘‘ uvek istinito. Šta ako tabela ,,Users‘‘ sadrzi imena i šifre? Haker moze pristupiti svim imenima i šiframa iz baze jednostavno dodavajuci ,,105 OR 1=1‘‘ u polje za unos korisničkog imena.

\paragraph{}
 ,, \texttt{"}\texttt{"}=\texttt{"}\texttt{"} ‘‘ je uvek istinito i ovo je još jedan način upotrebe SQL injekcije. Recimo da imamo sledeći deo koda:
\begin{verbatim}
uName = getRequestString("username");
uPass = getRequestString("userpassword");

sql = 'SELECT * FROM Users WHERE Name ="' + uName + '" AND 
								Pass ="' + uPass + '"'
\end{verbatim}

\noindent Haker jednostavno može pristupiti korisničkim imenima i šiframa u bazi unoseći ,, \texttt{"} OR \texttt{"}\texttt{"}=\texttt{"} ‘‘ u polje za šifru ili u polje za korisničko ime. Kod na serveru ce kreirati ispravan SQL upit:

\begin{verbatim}
SELECT * FROM Users WHERE Name ="" OR ""="" AND Pass ="" OR ""="";
\end{verbatim}

\noindent SQL upit koji se nalazi iznad vratiće sve redove iz tabele ,,Users‘‘, jer je ,, OR \texttt{"}\texttt{"}=\texttt{"}\texttt{"} ‘‘ uvek istinito. 

\paragraph{}
Mnoge baze podrzavaju grupu SQL upita razdvojene ,, ; ‘‘. SQL upit ispod vratiće sve redove iz tabele ,,Users‘‘, i potom obrisati ,,Suppliers‘‘ tabelu.

\begin{verbatim}
SELECT * FROM Users; DROP TABLE Suppliers
\end{verbatim}

\noindent Pogledajmo sledeći primer:

\begin{verbatim}
txtUserId = getRequestString("UserId");
txtSQL = "SELECT * FROM Users WHERE UserId = " + txtUserId;
\end{verbatim}

\noindent Ukoliko bi korisnik u polje za korisnički id uneo ,,105; DROP TABLE Suppliers‘‘ SQL upit koji se nalazi iznad izgledao bi ovako:

\begin{verbatim}
SELECT * FROM Users WHERE UserId = 105; DROP TABLE Suppliers; 
\end{verbatim}

\paragraph{}
Kako se zaštititi od ovakvih napada? Tako što cemo koristiti SQL parametre. SQL parametri su vredosti koje su dodate SQL upitu u vreme izvršavanja na kontrolisan način. 

\begin{verbatim}
txtUserId = getRequestString("UserId");
txtSQL = "SELECT * FROM Users WHERE UserId = @0";
db.Execute(txtSQL,txtUserId);
\end{verbatim}

\noindent Primer iznad je deo koda u ASP.NET-u u kome se koriste parametri. Parametri su predstavljeni znakom @. SQL mehanizam proverava parametre kako bi se uverio da su ispravni i da se tretiraju bukvalno a ne kao deo SQL-a koji se izvršava.


\subsection{DoS napadi}
\label{subsec:DoS}

https://www.paloaltonetworks.com/cyberpedia/what-is-a-denial-of-service-attack-dos

DoS (Denial-of-Service) napad je radnja napravljena tako da spreči legitimitne korisnike da koriste usluge računara, tj. počinilac (izvršilac) čini mašinu ili mrežni resurs nedostupnim (gasi ih) svojim korisnicima tako što privremeno ili neograničeno ometa usluge hosta povezanog na Internet. DoS napad moze da uključi neovlašćeni pristup jednom ili više kompjuterskih sistema, ali cilj napada nije krađa informacija, nego je cilj da poremeti sposobnost servera da odgovori na korisničke zahteve tako što poplavljuje metu saobraćajem (?? popravi) ili šalje informacije koje aktiviraju razne nezgode. Ometanje normalnog rada kopjuterskih usluga moze da proizvede značajnu stetu. Firma koja se bavi nekom vrstom prodaje putem Interneta može da izgubi posao. Vojsci može da se prekine komunikacija. Vladi ili nekoj neprofitnoj organizaciji može da se desi da ne može da prenese svoju poruku javnosti.

DoS napad je primer "asimetričnog" napada, u kome jedna osoba moze dosta da naškodi velikoj organizaciji. Pošto se terorističke organizacije specijalizuju za asimetrične napade, neki strahuju da će DoS napadi postati važan deo terorističkog oružja.

\subsubsection{DDoS napadi}
\label{subsubsec:DDoS}

Dodatni tip DoS napada je DDoS (Distributed Denial-of-Service) napad. Glavna razlika je u tome što meta nije napadnuta sa jedne lokacije, već sa više njih odjednom. Do DDoS napada se dolazi kada višestruki sistemi orkestriraju sinhronizovani DoS napad na jednu metu. Podela hostova koji određuju DDoS daje napadaču više prednosti:
\begin{itemize}
\item Napadač može iskoristiti veću količinu mašine (popravi) da izvrši ozbiljno razoran napad
\item Lokacija napada se teško određuje zbog slučajne podele napadačkih sistema
\item Teže je ugasiti više mašina nego jednu
\item Pravu napadačku partiju (attacking party - potrazi) je veoma teško identifikovati, jer se oni prikrivaju iza mnogih (uglavnom kompromitovanih) sistema
\end{itemize}

Mnoge sigurnosne tehnologije su razvile mehanizme za odbranu od mnogih vrsta DoS napada, ali, zbog jedinstvenih karakteristika, DDoS se jos uvek smatra ozbiljnom pretnjom.

\section{Sajber kriminal}
\label{sec:sajber_kriminal}

Ovde pišem tekst. 
Ovde pišem tekst. 
Ovde pišem tekst. 
Ovde pišem tekst. 

\subsection{Primeri kriminalnih napada}
\label{subsec:primeri_krimi_napada}

Ovde pišem tekst. 
Ovde pišem tekst. 
Ovde pišem tekst. 
Ovde pišem tekst. 
Ovde pišem tekst. 
Ovde pišem tekst. 

\section{Sajber napadi}
\label{sec:sajber_napadi}

Ovde pišem tekst. 
Ovde pišem tekst. 
Ovde pišem tekst. 
Ovde pišem tekst. 
Ovde pišem tekst. 

\subsection{Primeri napada}
\label{subsec:primeri_napada}

Ovde pišem tekst. 
Ovde pišem tekst. 
Ovde pišem tekst. 
Ovde pišem tekst. 
Ovde pišem tekst. 


\section{Glasanje preko interneta}
\label{sec:glasanje}

Ovde pišem tekst. 
Ovde pišem tekst. 
Ovde pišem tekst. 
Ovde pišem tekst. 
Ovde pišem tekst. 
Ovde pišem tekst. 
Ovde pišem tekst. 
Ovde pišem tekst. 
Ovde pišem tekst. 

\subsection{podnaslov}
\label{subsec:podnaslov}

Ovde pišem tekst. 
Ovde pišem tekst. 
Ovde pišem tekst. 
Ovde pišem tekst. 
Ovde pišem tekst.

\section{Zaključak}
\label{sec:zakljucak}

Ovde pišem zaključak. 
Ovde pišem zaključak. 
Ovde pišem zaključak. 
Ovde pišem zaključak. 
Ovde pišem zaključak. 
Ovde pišem zaključak. 
Ovde pišem zaključak. 
Ovde pišem zaključak. 
Ovde pišem zaključak. 
Ovde pišem zaključak. 
Ovde pišem zaključak. 
Ovde pišem zaključak.

 
\addcontentsline{toc}{section}{Literatura}

\appendix
\bibliography{seminarski} 
\bibliographystyle{plain}
\begin{enumerate}
\item \url{https://www.trendmicro.com/vinfo/us/security/definition/cybercriminals}
\item \url{https://cybersecurityventures.com/cybercrime-damages-6-trillion-by-2021/} 
\item \url{https://www.paloaltonetworks.com/cyberpedia/what-is-a-denial-of-service-attack-dos}
\item \url{https://www.w3schools.com/sql/sql_injection.asp}
\end{enumerate}

\end{document}
